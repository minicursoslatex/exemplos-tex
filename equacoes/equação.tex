\documentclass{article}
\usepackage[utf8]{inputenc}
\usepackage{eqnarray}
\usepackage{url}
\usepackage[brazil]{babel}
\title{Um simples exemplo de equações no \LaTeX}
\author{Rafael Rampim Soratto  \\
	Universidade Tecnológica Federal do Paraná, UTFPR,\\ Campo Mourão, PR, Brasil \\
	\and 
	\small{soratto@alunos.utfpr.edu.br}
}

\date{\today}

\begin{document}
	
	\maketitle
	\begin{abstract}
		Um breve exemplo sobre equações do segundo grau utilizando a fórmula de bhaskara.
	\end{abstract}
	
	\section{Equações do Segundo grau}
	
	\subsection{Introdução}
	 De acordo com \cite{uol} a fórmula de Bhaskara é um método resolutivo para equações do segundo grau utilizado para encontrar raízes a partir dos coeficientes da equação.	
	\subsection{Fórmula de Bhaskara}
	 De acordo com a Equação \ref{eq:segundo_grau}, equações do segundo grau são equações definidas por polinômios de grau 2.Toda equação do segundo grau, em sua forma normal, estará escrita da seguinte maneira:
	 % \usepackage{eqnarray}
	 \begin{eqnarray}
	  	ax^2+bx+c=0
	 	\label{eq:segundo_grau}
	 \end{eqnarray}
	A fórmula de \textit{Bhaskara} foi criada a partir do método de completar quadrados. Seguindo esse método para os coeficientes genéricos a, b e c, obtém-se a seguinte expressão: 
	
	$$x = \frac{-b\pm \sqrt[2]{b^2-4ac}}{2a}$$
		\label{eq:X}
	\begin{thebibliography}{9}
		\bibitem[UOL]{uol} \emph{Equações do segundo grau}, Dísponível em:
		\url{https://brasilescola.uol.com.br/matematica/formula-bhaskara.htm}. Acesso em março de 2020.
	\end{thebibliography}
	
\end{document}
