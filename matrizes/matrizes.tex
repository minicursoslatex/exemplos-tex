\documentclass{article}
\usepackage[utf8]{inputenc}
\usepackage{eqnarray}
\usepackage{url}
\usepackage[brazil]{babel}
\title{Um simples exemplo de equações no \LaTeX}
\author{Rafael Rampim Soratto  \\
	Universidade Tecnológica Federal do Paraná, UTFPR,\\ Campo Mourão, PR, Brasil \\
	\and 
	\small{soratto@alunos.utfpr.edu.br}
}

\date{\today}

\begin{document}
    \section{Matrizes}
    $$
        A=(a_{ij})_{3\times 3}=\left(\begin{array}{ccc}
        1& 2 & 3 \\
        4& 5 & 6\\
        7& 8 & 9
        \end{array} \right)
    $$

	\section{Integrais e Somatórios}
	\subsection{Integral}
	
	Uma equação utilizando integral: 
	
	\begin{eqnarray}
    X(s) = \int\limits_{t = -\infty}^{\infty} x(t) e^{-st}
    dt x^2
    \label{eq:Xs}
    \end{eqnarray}

    \subsection{Somatório}
    
    \begin{equation}
     e(t) = 
     \sum_{n=1}^{5} \frac{1}{2+n} \cos(2 \pi nt)
    \end{equation}

\end{document}

