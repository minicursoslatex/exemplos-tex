\documentclass{article}
\usepackage[utf8]{inputenc}
\usepackage{url}
\usepackage{float}
\usepackage[brazil]{babel}
\title{Um simples exemplo de equações no \LaTeX}
\author{Rafael Rampim Soratto  \\
	Universidade Tecnológica Federal do Paraná, UTFPR,\\ Campo Mourão, PR, Brasil \\
	\and 
	\small{soratto@alunos.utfpr.edu.br}
}
\date{\today}

\begin{document}
    \section{Tabelas}

\begin{table}[H]
\caption{Nome Exemplo} 
\label{tab:exemplo} 
\begin{center}
\begin{tabular}{|c|c|c|} 
\hline
CPF & Idade & Salario  \\
xxx.xxx.xxx-yy & 20 & 2.000 R\$ \\ \hline
xxx.xxx.xxx-yy & 22 & 2.002 R\$ \\ 
\hline
\end{tabular} 
\end{center}
\end{table}

\section{Quadros}
\begin{itemize}
\item 
	\begin{center}
		
\label{quadro:apelido} 
\begin{tabular}{|c|c|}  

\hline
a  & b \\   
\hline
c  & d \\
\hline 
\end{tabular}

\end{center}
\item 
\begin{center}
	\begin{minipage}{1\linewidth}  %minipage de espaco 1
		\centering
		\begin{tabular}{||c||c||}  
			\hline 
			\begin{tabular}{c} 
				Texto 1 
			\end{tabular} 
			& 
			\begin{tabular}{c} 
				Texto 2 
			\end{tabular}  \\   % //quebra as linhas
			
			\begin{tabular}{c} 
				Texto 3 
			\end{tabular} 
			& 
			\begin{tabular}{c} 
				Texto 4 
			\end{tabular} 
			\\  
			\hline   
		\end{tabular}  
	\end{minipage} 
\end{center} 

\end{itemize}
\end{document}

