\documentclass{article}
\usepackage[utf8]{inputenc}
\usepackage{eqnarray}
\usepackage{listings}
\usepackage{url}
\usepackage[brazil]{babel}
\title{Um simples exemplo de equações no \LaTeX}
\author{Rafael Rampim Soratto  \\
	Universidade Tecnológica Federal do Paraná, UTFPR,\\ Campo Mourão, PR, Brasil \\
	\and 
	\small{soratto@alunos.utfpr.edu.br}
}

\date{\today}

\begin{document}
	
	\maketitle
	\begin{abstract}
		Um breve exemplo sobre limites e derivadas
	\end{abstract}
 
\subsubsection{Limites e Derivadas}     

Para inserir no texto um limite, basta utilizar um comando do tipo: 
\begin{lstlisting}
\begin{itemize}
\item $\lim\limits_{x \to a} f(x)=f(a)$;
\item $\lim_{x \to a} f(x)=f(a)$;
\item $\displaystyle\lim_{x \to a} f(x)=f(a)$
\item $$\lim_{x \to a} f(x)=f(a)$$
\end{itemize}
\end{lstlisting} 
\textbf{Resultado}: 

\begin{itemize}
	\item $\lim\limits_{x \to a} f(x)=f(a)$;
	\item $\lim_{x \to a} f(x)=f(a)$;
	\item $\displaystyle\lim_{x \to a} f(x)=f(a)$
	\item $$\lim_{x \to a} f(x)=f(a)$$
\end{itemize}

Derivadas podem ser denotadas por 
\begin{lstlisting}
f^{\prime}
\end{lstlisting}

\textbf{Resultado}: 

\begin{enumerate}
	\item $f^{\prime}(x)$;
	\item $f^{\prime\prime}$;
	\item $f''$
	\item $y^{(5)}+y^{\prime\prime}$;
	\item $e$
	\item $\dot{z}$
	\item $\frac{d^3y}{dx^3}$;
	\item $\frac{\partial f}{\partial x}$;
	\item $\frac{\partial^2 f}{\partial x^2}$;
	\item $\frac{\partial^2 f}{\partial x\partial y}$;
	\item $\frac{\partial f}{\partial x}(a,b)=\lim\limits_{h \to 0} \frac{f(a+h,b)-f(a,b)}{h}$;
\end{enumerate} 


	
\end{document}
